%-----------------------------------------------------------------------------------------------------------------------------------------------%
%	The MIT License (MIT)
%
%	Copyright (c) 2021 Jitin Nair
%
%	Permission is hereby granted, free of charge, to any person obtaining a copy
%	of this software and associated documentation files (the "Software"), to deal
%	in the Software without restriction, including without limitation the rights
%	to use, copy, modify, merge, publish, distribute, sublicense, and/or sell
%	copies of the Software, and to permit persons to whom the Software is
%	furnished to do so, subject to the following conditions:
%	
%	THE SOFTWARE IS PROVIDED "AS IS", WITHOUT WARRANTY OF ANY KIND, EXPRESS OR
%	IMPLIED, INCLUDING BUT NOT LIMITED TO THE WARRANTIES OF MERCHANTABILITY,
%	FITNESS FOR A PARTICULAR PURPOSE AND NONINFRINGEMENT. IN NO EVENT SHALL THE
%	AUTHORS OR COPYRIGHT HOLDERS BE LIABLE FOR ANY CLAIM, DAMAGES OR OTHER
%	LIABILITY, WHETHER IN AN ACTION OF CONTRACT, TORT OR OTHERWISE, ARISING FROM,
%	OUT OF OR IN CONNECTION WITH THE SOFTWARE OR THE USE OR OTHER DEALINGS IN
%	THE SOFTWARE.
%	
%
%-----------------------------------------------------------------------------------------------------------------------------------------------%

%----------------------------------------------------------------------------------------
%	DOCUMENT DEFINITION
%----------------------------------------------------------------------------------------

% article class because we want to fully customize the page and not use a cv template
\documentclass[a4paper,12pt]{article}

%----------------------------------------------------------------------------------------
%	FONT
%----------------------------------------------------------------------------------------

% % fontspec allows you to use TTF/OTF fonts directly
% \usepackage{fontspec}
% \defaultfontfeatures{Ligatures=TeX}

% % modified for ShareLaTeX use
% \setmainfont[
% SmallCapsFont = Fontin-SmallCaps.otf,
% BoldFont = Fontin-Bold.otf,
% ItalicFont = Fontin-Italic.otf
% ]
% {Fontin.otf}

%----------------------------------------------------------------------------------------
%	PACKAGES
%----------------------------------------------------------------------------------------
\usepackage{url}
\usepackage{parskip} 	

%other packages for formatting
\RequirePackage{color}
\RequirePackage{graphicx}
\usepackage[usenames,dvipsnames]{xcolor}
\usepackage[scale=0.9]{geometry}
\usepackage{mlmodern}
%tabularx environment
\usepackage{tabularx}

%for lists within experience section
\usepackage{enumitem}

% centered version of 'X' col. type
\newcolumntype{C}{>{\centering\arraybackslash}X} 

%to prevent spillover of tabular into next pages
\usepackage{supertabular}
\usepackage{tabularx}
\newlength{\fullcollw}
\setlength{\fullcollw}{0.47\textwidth}

%custom \section
\usepackage{titlesec}				
\usepackage{multicol}
\usepackage{multirow}

%CV Sections inspired by: 
%http://stefano.italians.nl/archives/26
\titleformat{\section}{\Large\scshape\raggedright}{}{0em}{}[\titlerule]
\titlespacing{\section}{0pt}{10pt}{10pt}

%for publications
\usepackage[natbib, style=authoryear, sorting=ydnt, maxbibnames=2]{biblatex}

%Setup hyperref package, and colours for links
\usepackage[unicode, draft=false]{hyperref}
\definecolor{linkcolour}{rgb}{0,0.2,0.6}
\hypersetup{colorlinks,breaklinks,urlcolor=linkcolour,linkcolor=linkcolour}
\addbibresource{bib.bib}
\setlength\bibitemsep{1em}
\hypersetup{citecolor=linkcolour}

%for social icons
\usepackage{fontawesome5}

%debug page outer frames
%\usepackage{showframe}

%----------------------------------------------------------------------------------------
%	BEGIN DOCUMENT
%----------------------------------------------------------------------------------------
\begin{document}

% non-numbered pages
\pagestyle{empty} 

%----------------------------------------------------------------------------------------
%	TITLE
%----------------------------------------------------------------------------------------

% \begin{tabularx}{\linewidth}{ @{}X X@{} }
% \huge{Your Name}\vspace{2pt} & \hfill \emoji{incoming-envelope} email@email.com \\
% \raisebox{-0.05\height}\faGithub\ username \ | \
% \raisebox{-0.00\height}\faLinkedin\ username \ | \ \raisebox{-0.05\height}\faGlobe \ mysite.com  & \hfill \emoji{calling} number
% \end{tabularx}

\begin{tabularx}{\linewidth}{@{} C @{}}
\Huge{Yue Niu} \\[7.5pt]
\href{https://yuesforest.com}{\raisebox{-0.05\height}\faGlobe \ yuesforest.com} \ $|$ \ 
\href{https://github.com/kaonn}{\raisebox{-0.05\height}\faGithub\ github.com/kaonn} \ $|$ \ 
% \href{https://linkedin.com/in/username}{\raisebox{-0.05\height}\faLinkedin\ username} \ $|$ \ 
% \href{mailto:yuen@andrew.cmu.edu}{\raisebox{-0.05\height}\faEnvelope \ yuen@andrew.cmu.edu} \ 
\href{mailto:yueyueniu@gmail.com}{\raisebox{-0.05\height}\faEnvelope \ yueyueniu@gmail.com} \ 
% \href{tel:+000000000000}{\raisebox{-0.05\height}\faMobile \ +00.00.000.000} \\
\end{tabularx}

%----------------------------------------------------------------------------------------
% EXPERIENCE SECTIONS
%----------------------------------------------------------------------------------------

%Interests/ Keywords/ Summary
% \section{Summary}

% My work is centered around ergonomic \emph{logical frameworks} for studying 
% \textbf{cost} and \textbf{behavior} of functional programs from both the 
% \emph{local} and \emph{global} perspective. 

% The local perspective is the analysis of individual programs. 
% There are many well-developed settings for studying the behavioral properties of \emph{functional} 
% programs alone; however the combination of \emph{cost} and behavior has traditionally been 
% difficult to study in a compositional and faithful manner. My work contributes a 
% formal setting in which one may speak about cost-aware properties of programs while retaining 
% desirable mathematical properties such as functional extensionality. 
% My collaborators and I have implemented the resulting logical framework and 
% formalized many common pen-and-paper patterns of cost analysis. 

% The global perspective is study of the overall structure of programming languages. 
% One direction is the generalization of classic denotational semantics 
% to the cost-aware setting. A natural criterion is the cost-aware generalization of 
% \emph{adequacy}: the (cost-aware) denotational and operational meaning 
% of programs should coincide. Therefore, the notion of adequacy can be used to 
% ``ground'' the theorems inside the logical framework to the actual behavior \emph{and cost} of programs. 
% By viewing programs as \emph{internal} mathematical constructions inside logical frameworks, 
% I have developed this idea for an \emph{Algol}-like language as a basis for validating cost models. 


%----------------------------------------------------------------------------------------
%	EDUCATION
%----------------------------------------------------------------------------------------
\section{Education}
\begin{tabularx}{\linewidth}{@{}l X@{}}	
2018 - present & Ph.D. in Computer Science at \textbf{Carnegie Mellon University} \hfill \normalsize \\

2014 - 2017 & B.S. in Computer Science, Minor in Physics at \textbf{Carnegie Mellon University} \hfill 
\end{tabularx}

%Experience
\section{Academic positions}

\begin{tabularx}{\linewidth}{ @{}l X@{} }
  Aug 2018 - present & \textbf{Doctoral student}, advised by Prof. Robert Harper \\[3.75pt]
  Topic: & \emph{Logical frameworks for cost analysis}\\
  Publications: &\citet{niu-harper:2023,niu-sterling-grodin-harper:2022,grodin-niu-sterling-harper-2023}\\ \\ 
  Jan 2018 - July 2018 & \textbf{Research assistant}, advised by Prof. Jan Hoffmann \\[3.75pt]
  Topic: & \emph{Static analysis for functional programs with garbage collection}\\ 
  Publications: & \citet{niu-hoffmann:2018}
\end{tabularx}

\section{Fellowships}

\begin{tabularx}{\linewidth}{ @{}l X@{} }
  Aug 2020 - Aug 2023 & 2020 National Defense Science and Engineering Graduate (NDSEG) Fellowship Award
\end{tabularx}

\section{Work experience}
\begin{tabularx}{\linewidth}{ @{}l X@{} }
  May - Aug 2016 & \textbf{Software dev. intern}, \emph{Viavi Solutions} \\[3.75pt]
  Project: & \emph{Accelerated data-plane packet receiver} \\ \\ 
  May - Aug 2015 & \textbf{Software dev. intern}, \emph{Testive} \\[3.75pt]
  Project: & \emph{Parsing pictures of Scantron sheets} 
%   \multicolumn{2}{@{}X@{}}{
%   \begin{minipage}[t]{\linewidth}
%       \begin{itemize}[nosep,after=\strut, leftmargin=1em, itemsep=3pt]
% \item[--] Designed an accelerated data-plane packet receiver using Intel's Data Plane Dev. Kit (DPDK) 
% \item[--] Engineered a modular architecture using C++11 around the C API
% \item[--] Increased receiving throughput from 2-3Gbps to 10Gbps
% \item[--] Ensured Inter-operability with the Linux kernel networking stack
%       \end{itemize}
%   \end{minipage}
%   }
\end{tabularx}

% \begin{tabularx}{\linewidth}{ @{}l r@{} }
%   \textbf{Software Dev. Intern} &\hfill May - Aug 2016 \\[3.75pt]
% \emph{Testive} &Boston, MA\\
%   \multicolumn{2}{@{}X@{}}{
%   \begin{minipage}[t]{\linewidth}
%       \begin{itemize}[nosep,after=\strut, leftmargin=1em, itemsep=3pt]
% \item[--] Collaborated with other interns to develop a cross platform mobile application
% \item[--] Coordinated development of core functionality
% \item[--] Employed the OpenCV (computer vision) API to parse image files into user input
% \item[--] Integrated backend logic into existing Django code base
%       \end{itemize}
%   \end{minipage}
%   }
% \end{tabularx}


%Projects
\section{Projects}

\begin{tabularx}{\linewidth}{ @{}l r@{} }
\textbf{calf: a \textbf{c}alf-\textbf{a}ware \textbf{l}ogical \textbf{f}ramework} & \hfill \href{https://github.com/jonsterling/agda-calf}{Project link}
\end{tabularx}

\section{Teaching assistant} 

\begin{tabularx}{\linewidth}{ @{}l l@{} }
  Fall 2022, Types and Programming languages & \emph{with Prof. Jan Hoffmann}\\
  Fall 2020, Foundations of Programming languages & \emph{with Prof. Robert Harper} 
\end{tabularx}



%----------------------------------------------------------------------------------------
%	PUBLICATIONS
%----------------------------------------------------------------------------------------
\section{Publications}
\begin{refsection}[bib.bib]
\nocite{*}
\printbibliography[heading=none]
\end{refsection}

\section{Manuscripts}
\begin{refsection}[manu.bib]
\nocite{*}
\printbibliography[heading=none]
\end{refsection}
%----------------------------------------------------------------------------------------
%	SKILLS
%----------------------------------------------------------------------------------------
% \section{Skills}
% \begin{tabularx}{\linewidth}{@{}l X@{}}
% Some Skills &  \normalsize{This, That, Some of this and that etc.}\\
% Some More Skills  &  \normalsize{Also some more of this, Some more that, And some of this and that etc.}\\  
% \end{tabularx}

\vfill
\center{\footnotesize Last updated: \today}

\end{document}
